\documentclass[12pt]{ctexart}
\usepackage{extsizes}
\usepackage{amsmath}
\usepackage{geometry}
\usepackage{graphicx}
\usepackage{geometry}
\usepackage{color}
\usepackage{fancyhdr}
\usepackage{threeparttable}
\usepackage{float}
\geometry{a4paper, scale=0.8}
\graphicspath{ {./image/} }
\definecolor{ngrey}{RGB}{77, 93, 83}
\pagestyle{fancy}
\fancyhead{}
\fancyhead[C]{同化棋实验报告\quad 梁博强}
\setcounter{section}{-1}

\begin{document}
	\title{同化棋: 一种基于蒙特卡罗树搜索的解决方案}
	\author{梁博强\quad 元培学院}
	\date{\today}
	\maketitle

	\begin{abstract}
		\normalsize
		在本学期计算概论课程的同化棋大作业中,笔者采用了一种基于蒙特卡罗树搜索(MCTS)的算法来实现
		并优化机器决策,并针对同化棋进行改进,最终取得了优于朴素贪心算法的成绩。
		本文将简要介绍这次大作业的设计思路、实现过程、决策优化、相关工作
		和后续进一步工作的方向。
		
		通过本次大作业的实践,我对各种主流的博弈算法有了初步了解。此外笔者将C++实现的决策算法
		封装在动态链接库中,并搭建了与Python编写的图形界面框架的连接,这加深了我对程序之间交互的
		理解。而在实现程序具体功能、提高程序实用性的过程中,笔者对文件读写、面向对象编程、树数据结构、异常
		处理等有了进一步认识。
	\end{abstract}

	\section{背景}

	\subsection{同化棋}
	同化棋(Ataxx),是Dave Crummack和Craig Galley在1988年发明,1990年出品于电视游戏而流行的两人棋类,可说是黑白棋的衍生。
	其最主要的特点是落子后会将邻近八格的所有敌方棋子颜色翻转,并且原棋子有可能是被复制的,也有可能是被移动的。

	同化棋的上述特点为机器决策带来了难度: 相比于相同规模的其他常见棋类,从当前局面状态到下一局面状态的\textbf{决策分支}可能更多;当前状态与
	下一状态间的\textbf{局面估值}(value)的差可能较大,这就要求我们在蒙特卡罗树搜索算法的基础上加以改进。

	\subsection{蒙特卡罗树搜索}

	\section{不足与后续工作}
	必须承认,笔者完成这份作业较为仓促,因此仍有很多地方有待改进。

	\subsection{MCTS探索节点数较少}
	尽管笔者尝试采用限制展开环节的深度,以及用叶节点的局面的估值(棋子数之差),但在规定时间内所能
	探索的节点数仍然较少,仅达到1000个左右。而同化棋中,每个状态节点对应的子节点约有10-100个,MCTS要求我们完成展开的次数应明显高于
	每个节点的子节点数,才能实现较好的决策。

	反映在结果中,我们可以看到本算法在对局开始(前15回合)和结尾(后15回合)时的决策明显优于中期的决策。
	原因可能是在对局前期,决策算法会倾向于先复制并占领更多的空白,或是试图在搜索深度限制内吃掉对方所有棋子。
	在对局末期,MCTS几乎能枚举到所有最可能被打出的决策,从而给出精确的最优决策。但在对局中期,MCTS很难模拟展开到终局,
	使得节点的估值很不精确。

	探索节点数较少的原因可能在于: 首先,同化棋的决策分支数相对较多。更重要的是,采用贪心算法,而非随机决策进行展开模拟虽然可以获得比较好的模拟结果,但是对于每一个局面
	我们需要枚举所有可能的落子决策,这极大增加了算法的时间复杂度。

	后续我们可以进一步优化不同回合时MCTS展开的深度限制,使得我们在“探索更多节点”与“模拟到更多终局”间取得一个平衡。另外,我们可以优化展开时的贪心算法,从所有移动方法为“复制”
	的落子决策中取得最优解;抑或是用叶节点的局面估值代替模拟展开结果,这样都能减少模拟展开时的时间复杂度。
		

	\subsection{搜索树未被长时保留}
	在 Botzone 平台上运行以及线下提交的版本中,笔者采用的都是常见的短时运行方式,即每一次决策结束后退出搜索
	函数,同时清空搜索树。这样做的稳定性较高,但每一次决策时都要重新构建搜索树。有趣的是,MCTS最终选择的子节点是我们探索次数最多的
	节点,往往有最多已探索的后代节点,重新探索、模拟、展开这些后代节点是不必要的。

	保留搜索树、采用长时运行模式并不难,但关键是决策后需要清理其他的分支节点,避免造成内存浪费。否则,经过数轮决策,搜索树
	的大小会显著超过Botzone上的内存限制。也有必要优化数据结构,减小单个节点占用的空间。
	
\end{document}